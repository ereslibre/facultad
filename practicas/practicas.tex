\documentclass[12pt]{article}
\usepackage[utf8]{inputenc}
\title{Realización de Prácticas Formativas}
\author{Rafael Fernández López\\Ingeniería en Informática}
\date{Septiembre 2011}
\begin{document}
  \maketitle
  \section{Proyecto}
    El proyecto está enmarcado dentro del equipo de desarrollo de Interdominios. Las tareas a
    realizar son las siguientes:
    \begin{itemize}
      \item Creación de una extranet para los clientes
      \item Desarrollo de plataforma Bluetooth propia para KDE
    \end{itemize}
    El modelo de desarrollo de todos los subproyectos es TDD (Test Driven Development), de manera
    que conseguimos desarrollar un producto estable, así como asegurar que en futuras mejoras y
    ampliaciones las funcionalidades implementadas continuarán funcionando de forma óptima.
  \section{Creación de una extranet para los clientes}
    Ha sido necesario crear una extranet para los clientes que cuentan con muchos productos para
    permitirles hacer operaciones en modo "batch" masivo. Esto ayuda a los clientes que tienen
    muchos productos aprovechar mejor el tiempo requerido a la administración de sus productos.
    Las tecnologías utilizadas para el desarrollo de este panel han sido las siguientes:
    \begin{itemize}
      \item PHP
      \item Ruby
      \item C/C++
      \item Qooxdoo
      \item MySQL
    \end{itemize}
    Al comienzo de las prácticas el panel era una idea con unos casos de uso bien especificados.
    Tras los tres meses de prácticas y con el trabajo de todo el equipo y ayuda de mis tutores se
    ha conseguido crear un sistema completamente nuevo con un diseño modular.
    El nuevo panel permite registrar, renovar, trasladar y realizar todo tipo de acciones sobre
    dominios y sus contactos. Además, y por haber desarrollado una base genérica y extensible
    también se han desarrollado a posteriori otras aplicaciones:
    \begin{itemize}
      \item Aplicación que permite crear, modificar y gestionar cuentas de correo electrónico desde
            el punto de vista del administrador (cuotas...)
      \item Aplicación de monitorización de máquinas virtuales, que también permite a los clientes
            comprobar el estado de sus máquinas, así como reiniciar, apagar e iniciar las máquinas
    \end{itemize}
    Gracias a que la plataforma desarrollada es agnóstica, genérica y modular se puede extender su
    uso a cualquier tipo de proyecto orientado tanto al cliente final como a los administradores de
    sistemas. Únicamente es requerida la creación de una interfaz de usuario en el lenguaje y
    framework deseado, contactando con el servidor en un formato conocido (JSON, XML), creando de
    esta forma aplicaciones cliente-servidor; siendo el servidor el que dentro de la plataforma
    contiene toda la lógica de negocio.
    \subsection{Seguridad}
      Se ha prestado especial atención a la seguridad, por el hecho de ser un proyecto clave dentro
      de la estrategia interna. En este caso el sistema es complejo y está compuesto por tres
      servidores, siendo uno de ellos el único que es accesible desde fuera de la red interna.
      De esta manera, el servidor que contiene el cliente web es el único accesible por el cliente.
      La interfaz web realizará ciertas acciones que serán reportadas mediante un formato propio al
      segundo servidor, mediante red privada interna y encriptada. Este servidor se encargará de
      modificar las bases de datos necesarias, así como de incluir trabajos que requieran de contacto
      con proveedores de dominios, que serán incluidos en el tercer y último servidor, encargado
      de contactar con los proveedores.
    \subsection{Migración}
      Ha sido necesario desarrollar una herramienta de migración de usuarios actuales al nuevo
      sistema que les permitirá realizar operaciones en modo "batch".
      \subsubsection{Plan de migración}
        Debido a la gran cantidad de clientes con la que cuenta Interdominios, ha sido necesario
        realizar un plan de migración. Hemos contactado con diversos clientes que eran propicios para
        el uso de este sistema, migrando a un número limitado de clientes con grandes cuentas. Además,
        y tras comprobar que el sistema estaba funcionando correctamente para ellos, también se abrió
        el sistema para los nuevos clientes. Tras unos meses con los nuevos clientes utilizando el
        nuevo sistema, así como los clientes elegidos para realizar pruebas, se procederá a la migración
        completa de todos los clientes.
\end{document}
