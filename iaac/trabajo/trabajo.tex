\documentclass[12pt]{article}
\usepackage[spanish]{babel}
\usepackage{amsmath}
\usepackage{amssymb}
\usepackage[utf8]{inputenc}
\setlength{\topmargin}{-3.5cm}
\setlength{\footskip}{2cm}
\setlength{\oddsidemargin}{0cm}
\setlength{\evensidemargin}{0cm}
\setlength{\textwidth}{6.5in}
\setlength{\textheight}{11in}
\title{Modelo dinámico de un sistema físico\\Velocidad de crucero de un coche}
\author{Rafael Fernández López}
\date{}

\pagestyle{empty}

\begin{document}
\maketitle

Muchos coches en la actualidad cuentan con un sistema de ``velocidad de crucero'', que consiste en
mantener el coche con una velocidad constante sin necesidad de apretar el acelerador. A continuación
haré un diseño y análisis de un posible modelo dinámico para este sistema.

Como en todo sistema medianamente complejo actúan un gran número de variables. En este caso se
adoptará un enfoque sencillo, teniendo en cuenta varios factores determinantes:

\begin{itemize}
  \item $v$: velocidad real del coche
  \item $v_r$: velocidad deseada del coche
  \item $u$: salida de nuestro modelo. Controla el acelerador
  \item $T$: par de torsión. Parte que modela el comportamiento del motor
  \item $F$: fuerza generada por el par de torsión, aplicada finalmente a las ruedas del coche
  \item $F_d$: fuerza desequilibrante, generada por desniveles en la carretera, rozamiento y presión
               aerodinámica
  \item $m$: masa total del coche
\end{itemize}

El tipo de controlador más adecuado sería un PI, ya que disminuirá el error en estado estacionario, y
tendrá noción del error acumulado hasta el momento, promediándolo. Sin embargo, no podemos añadir a
nuestro controlador un componente derivativo, ya que por la propia naturaleza del problema que queremos
resolver, no podemos saber como cambiarán las variables de estado, esto es, en gran medida: la fuerza
desequilibrante, marcada principalmente por los desniveles en la carretera.

Una posible ecuación para el movimiento del coche podría ser la siguiente:

\begin{equation}
  m \dfrac{dv}{dt} = F - F_d
\end{equation}

La fuerza generada por el par de torsión podría estar representada por la siguiente ecuación:

\begin{equation}
  T(\omega) = T_m \left(1 - \beta \left(\dfrac{\omega}{\omega_m}-1\right)^2\right)
\end{equation}

La máxima fuerza que puede ejercer el par de torsión se alcanza cuando $w=w_m$, es decir, cuando la velocidad
angular alcanza su máximo. $\beta$ es un factor de corrección para aplicar más realidad al modelo.

La velocidad del motor está relacionada con la velocidad del coche por la expresión:

\begin{equation}
  \omega = \dfrac{n}{r} v = \alpha_n v
\end{equation}

donde $n$ representa la relación de transmisión y $r$ el radio de las ruedas.

De tal forma, que la fuerza aplicada a la conducción sería:

\begin{equation}
  F = \dfrac{nu}{r} T(\omega) = \alpha_n u T(\alpha_n v)
\end{equation}

Las fuerzas desequilibrantes ($F_d$) son tres principalmente: $F_g$, que representa la fuerza que ejerce
la gravedad, $F_r$ que representa la fricción y $F_a$ que representa la resistencia aerodinámica.

La fuerza $F_g$ ejercida por la gravedad se podría representar así:

\begin{equation}
  F_g = m g sen(\theta)
\end{equation}

donde $\theta$ es el ángulo de desnivel de la carretera.

Un posible modelo para la fuerza de rozamiento podría ser el siguiente:

\begin{equation}
  F_r = m g C_r
\end{equation}

donde $C_r$ es el coeficiente de fricción, dependiente del tipo de carretera que estemos atravesando.

Por último, la fuerza de la resistencia aerodinámica se puede representar de la siguiente forma:

\begin{equation}
  F_a = \dfrac{1}{2} \rho C_d A v^2
\end{equation}

donde $\rho$ es la densidad del aire, $C_d$ es el coeficiente aerodinámico dependiente de la forma
del coche, y $A$ es la sección frontal del coche.

Por tanto, la ecuación completa para nuestro controlador de la velocidad de crucero del coche
quedaría:

\begin{equation}
  m \dfrac{dv(t)}{dt} = \alpha_n u T(a_n v(t)) - m g C_r - \dfrac{1}{2} \rho C_d A v^2 - m g sen(\theta)
\end{equation}

Aplicando la transformada de Laplace:

\begin{equation}
  m s V(s) = \alpha_n u T(a_n V(s)) V(s) - m g C_r V(s) - \dfrac{1}{2} \rho C_d A v^2 V(s) - m g sen(\theta) V(s)
\end{equation}

\hfill

\begin{flushright}
  $\blacksquare$
\end{flushright}

\end{document}
